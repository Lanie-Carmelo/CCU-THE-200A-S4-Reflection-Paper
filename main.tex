\documentclass[stu,12pt,floatsintext]{apa7}

% Language and citation setup
\usepackage[american]{babel}
\usepackage{csquotes}
\usepackage[style=apa,sortcites=true,sorting=nyt,backend=biber]{biblatex}
\DeclareLanguageMapping{american}{american-apa}
\addbibresource{references.bib}

% Font and encoding
\usepackage{fontspec}
\setmainfont{Times New Roman}
\usepackage{unicode-math}
\setmathfont{Latin Modern Math} % or another LuaLaTeX-compatible math font

% PDF tagging for accessibility
\hypersetup{
  pdftitle={Session 4 Reflection},
  pdfauthor={Lanie Molinar},
  pdfsubject={Introduction to Systematic Theology (THE-200A)},
  pdfkeywords={Systematic Theology, Reflection, APA Style, Session 4},
}

% Document metadata
\title{Session 4 Reflection}
\author{Lanie Molinar}
\authorsaffiliations{California Christian University}
\duedate{July 14, 2025}
\course{Introduction to Systematic Theology (THE-200A)}
\professor{Dr. Cari Nimeth}

\begin{document}

\maketitle
\thispagestyle{plain}
\pagestyle{plain}

\section{How is the atonement to be understood in light of the other doctrines of the Christian faith?}

According to \textcite[p. 276]{ericksonIntroducingChristianDoctrine2015}, the atonement is "a crucial point of Christian faith, because it is the point of transition, as it were, from the objective to the subjective aspects of Christian theology." This means that the atonement is not just a historical event but also a theological concept that connects various doctrines such as sin, salvation, and grace. The atonement provides the foundation for understanding how God reconciles humanity to Himself through Jesus Christ, which is essential for grasping the full scope of Christian theology. It is through the atonement that we see the fulfillment of God's promises, the demonstration of His love, and the means by which believers can experience forgiveness and new life. The atonement also informs our understanding of God's justice and mercy, as it reveals how God can be both just in punishing sin and merciful in providing a way for sinners to be redeemed. Thus, the atonement serves as a pivotal doctrine that integrates and illuminates other key aspects of Christian faith.

\section{What elements are involved in the basic meaning of the atonement and why?}

The elements involved in the basic meaning of the atonement include sacrifice, propitiation, substitution, and reconciliation \parencite[pp. 288-289] {ericksonIntroducingChristianDoctrine2015}. Sacrifice refers to the offering of Jesus' life as a payment for sin, fulfilling the Old Testament sacrificial system. Propitiation involves appeasing God's wrath against sin, while substitution emphasizes that Jesus took our place, bearing the penalty we deserved. Reconciliation signifies the restoration of the relationship between God and humanity that was broken by sin. These elements are crucial because they illustrate the multifaceted nature of the atonement and its comprehensive role in salvation. Together, they demonstrate how Jesus' death and resurrection address the problem of sin and restore humanity's relationship with God.

\section{What significance for Christian theology may be drawn from the penal substitution theory of the atonement?}

The penal substitution theory, also known as the satisfaction theory, posits that Jesus's death was a substitutionary sacrifice that satisfied God's justice by bearing the penalty for human sin \parencite[pp. 279-280]{ericksonIntroducingChristianDoctrine2015}. It not only covered our sin so God no longer saw it, but it calmed His wrath and satisfied His justice. This theory is significant for Christian theology as it underscores the seriousness of sin and the necessity of divine justice. It affirms that God's love and justice are not in conflict; rather, they are harmonized in the atonement. By accepting this theory, Christians can better understand the depth of God's grace and the cost of salvation, leading to a more profound appreciation of Christ's sacrifice. Furthermore, it emphasizes the personal nature of salvation, as each believer is seen as having a direct relationship with Christ's atoning work. This theory also provides a framework for understanding the transformative power of the atonement in the life of a Christian, encouraging a response of faith and obedience. It differs from other theories, such as the moral influence theory, by focusing on the legal and judicial aspects of atonement rather than merely the moral example set by Christ. Thus, penal substitution remains a central tenet of orthodox Christian theology, shaping the understanding of salvation and the nature of God's relationship with humanity.

\textit{Taken together, these insights reveal that the doctrine of atonement is not an isolated concept but a theological cornerstone that integrates the justice, mercy, and love of God.} Through its multifaceted elements and the lens of penal substitution, the atonement reveals how Christ’s sacrifice fulfills divine justice while restoring humanity’s relationship with God, making it central to the coherence and transformative power of Christian theology.

\section{References}

\printbibliography

\end{document}